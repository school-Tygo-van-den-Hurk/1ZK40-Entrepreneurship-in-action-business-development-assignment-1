\section{Gravity-Based Energy Storage for Solar \& Wind Energy Farms} 

    With an increased demand for renewable energy sources such as wind and solar, effective energy storage is essential due to their dependence on weather conditions. Without storage, excess energy could be wasted and 24/7 energy availability cannot be guaranteed. 
    
    Currently, solar and wind farms commonly use: 
    
    \begin{itemize} 
        \item \textbf{Lithium-Ion Batteries:} High energy density, long cycle life, fast charging, and compact installation. Drawbacks include high cost, limited lifespan, safety concerns, environmental impact, and temperature sensitivity. 
        \item \textbf{Thermal Energy Storage (TES):} Stores excess energy as heat in materials, making use of otherwise wasted energy. Benefits include high energy density; drawbacks are safety risks, environmental sensitivity, and heat losses during storage and use. 
    \end{itemize} 
    
    \textbf{Gravity-Based Energy Storage (GBES)} offers several advantages over these methods: 
    
    \begin{itemize} 
        \item Minimal environmental limitations and dependency on weather. 
        \item Safer and environmentally friendly, with no rare minerals or dangerous chemicals. 
        \item Low energy loss during storage, as it does not dissipate as heat. 
    \end{itemize} 
    
    Large-scale GBES facilities, such as the Energy Vault project in Rudong, China, demonstrate practical implementation. Target customers include municipalities and corporations managing large-scale renewable energy farms.
    
    \begin{figure}[h!] 
        \centering 
        \includegraphics[width=0.8\columnwidth]{assets/gg.png} 
        \label{fig:GBES} 
    \end{figure}
    
    This application has been analysed using MON Worksheet 2. 
    
    \begin{figure}[h!] 
        \centering 
        \includegraphics[width=0.8\columnwidth]{assets/mon-2.png} 
        \caption{Example of a gravity-based energy storage facility for renewable energy.} 
        \label{fig:GBES} 
    \end{figure} 
    
    \subsection{Attractiveness Assessment (MON Worksheet 2)}
        
        \subsubsection{Potential}
        
            \paragraph{Compelling Reason to Buy — \textit{Super High}.}  
                The need for a sustainable, large-scale energy storage technology is very high, and especially one that is as safe as GBES addresses an unmet need. GBES effectively solves existing problems and outperforms current solutions in both energy storage efficiency and minimal risk.
            
            \paragraph{Market Volume — \textit{High}.}  
                The market of local governments, municipalities, and energy corporations is already substantial. With the growing number of solar and wind farms being constructed, the demand for energy storage solutions is expected to increase continuously.
            
            \paragraph{Economic Viability — \textit{Mid}.}  
                Margins are moderate as large-scale facilities require significant investment, and government budgets are often constrained. Large corporations have more financial flexibility to adopt the technology.
        
        \subsubsection{Challenge}
        
            \paragraph{Implementation Obstacles — \textit{Mid}.}  
                The investment is large and construction periods are lengthy. However, materials are generally easy to source, and no complex distribution network is required.
            
            \paragraph{Time to Revenue — \textit{High}.}  
                Sales cycles are long since the facilities take years to become operational. Due to high upfront costs, it will take multiple years for the facility to return its investment.
            
            \paragraph{External Risks — \textit{Low}.}  
                External risks are minimal: the market is not saturated, dependencies outside construction and engineering are limited, and environmental constraints are reduced due to GBES’s versatility in site selection.
            
            \paragraph{Summary.}  
                The overall potential of this market opportunity is high, with medium-level implementation risks. When these risks are managed, GBES offers a viable and long-term revenue-generating solution for large-scale renewable energy storage.
            